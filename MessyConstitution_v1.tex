\documentclass[12pt]{report}
\usepackage{xcolor,hyperref,amsmath,amssymb,graphics,color,url,graphicx,multicol,setspace}
\usepackage{mathrsfs,centernot,amsthm,algorithmic,mathtools,wrapfig}
\usepackage[margin={3 cm, 3 cm, 2 cm,2 cm }]{geometry}%l,r,t,b margin widths
\usepackage[utf8]{inputenc}
\usepackage[english]{babel}
\usepackage{tikz,lipsum,lmodern}
\usepackage[most]{tcolorbox}
\usepackage{upgreek}
\usepackage{geometry}
\usepackage{graphicx,yfonts}
\usepackage[Lenny]{fncychap}
\definecolor{titlepagecolor}{cmyk}{1,.60,0.5,.40}
\definecolor{namecolor}{cmyk}{1,.50,0,.10} % Here I am trying to define the font color that is used in the example but I don't really know which one is it so I leave this to the OP to figure out

%-----------------------------------------------------------------
\begin{document}
% ----------------------------------------------------------------

\begin{titlepage}
\newgeometry{left=3.5cm} %defines the geometry for the titlepage
\pagecolor{titlepagecolor}
\noindent
\includegraphics[width=2cm]{logo.png}\\[-1em]
\color{white}
\makebox[0pt][l]{\rule{1.3\textwidth}{1pt}}
\par
\noindent
\textbf{\textsf{The Institute Of Mathematical Sciences,}} \textcolor{namecolor}{\textsf{Chennai}}
\vfill
\noindent

{\Huge $\mathcal{C}$onstitution \hspace{1pt} of\\}
 \par  
 {\Huge the Students' Mess}  
\vfill
\noindent
\textcolor{namecolor}{\huge \textsf{First Students' Mess Committee}}
\vskip\baselineskip
\noindent
\textsf{Jun 20 2021}
\end{titlepage}
\restoregeometry % restores the geometry
\nopagecolor% Use this to restore the color pages to white

\newpage

\tableofcontents


\chapter{Introduction}
%\input{}
\section{Objective}
This constitution is to provide a functioning guideline for maintaining the functioning of the student's mess. This would most likely be an evolving document. Even before starting off describing the core functionalities, we, the \textbf{registered customers} \ref{rc} of the \textbf{students' mess}, agree to the core characteristics that this document should comprise of. In that spirit, let's first attempt to formally describe the problem we are trying to address.
\par The crux of the \textbf{ Mess Problem} that we are trying to solve involves the following :
\begin{itemize}
\item We want to meet the \textit{needs} of our fellow registered customers in terms of food quality, quantity and price.
\item The \textit{needs} of our customers include not only a balanced nutrition and cost effective meals but also subjective aspects like taste.
\item For doing so we have to encounter with an evolving market place within some proximity of our institute. We understand the nature of inflation and the aspects of price rise in an economy and therefore, understand that our \textbf{upgradations} \ref{upg} should factor in such changes.
\item Moreover, we understand that we might not only have to solve the problem of meeting our registered customers demands but we also need to be to minimise \textbf{excess food wastage} as much as possible in our day to day functioning. That is we have to take care of a supply side issue. Balancing both competing aspects to the problem is the trick of the trade and we hope that each of us can bring to the table our expertise to get better \textbf{solutions} \ref{sol} in successive \textbf{upgradations}\ref{upg}.
\item To solve such a complicated optimisation problem we understand we have \textbf{employees} \ref{emp} employed to aid us in our functioning. 
\end{itemize}

\chapter{Nature of this Constitution}
Before laying out the guidelines, we also need to lay out the central attributes of this document. These characteristics should not be ever \textbf{modified}\ref{mod}.
\par In this regards the following should be central tenets of this document that we should preserve.
\section{Core Characteristics}
\begin{enumerate}
\item Acknowledge that we are trying to provide a solution to an evolving dynamical problem within the framework of some simple logical rules.  Thus, while upholding the constitution, we should always remember that we are fallible and can only do our best in our turns. 
\item  The evolving nature of this document must be preserved. That is in no future committee formation should we think of these \textbf{guidelines}\ref{gdl} as gospel truth. We should always understand them as general guidelines, a blueprint of sort to assist in our working out of the solution to the mess problem. 
\item The above point shouldn't be interpreted to mean that we can just as well stray away from the guidelines as and how we want. The committee members should always strive to \textbf{uphold the instance of constitution they come upon}. 
\item The mechanism for change or alterations can be initiated by the members of mess committee. They have to activate an \textbf{issue of concern} \ref{ioc} and get the changes ratified by a \textbf{consensus mechanism} \ref{cons}.
\item We should always keep the entire document as precise and unambiguous as possible. Formally define whatever is probably not understandable by all. We have to keep in mind that we might need to pass on our guidelines to the newer generation of people who wouldn't be acquainted with this. We should strive to always make each other have the same image/interpretation of the constitution as is laid out in the most recent update of the constitution.
\end{enumerate}
%If we ever have a pledging ceremony for the committee representatives ever in the future it is these values/characteristics that we should always strive to uphold.

\chapter{Guidelines \label{gdl}}
We need to be able to agree on the words that we use in this document. And keep it as precise and unambiguous as possible. Hence we need to provide definitions for words that everyone might not be familiar with. Perhaps the best way to do this would be, to first come up with the guidelines and then be aware of the ideas we are smuggling in to assert those guidelines and provide handy definitions to those ideas. 
\section{Definitions}
\begin{itemize}
\item{\textbf{Mess Problem}} The \textbf{problem} elucidated in chapter 1.
\item{\textbf{Mess Committee}} A group of \textbf{at least 3} and \textbf{at most 5} textbf{IMSc students} working on the Mess Problem. 
 \item{\textbf{Employees} \label{emp}} The cooks and cleaning staff that aid us in the functioning of the student's mess.
\item{\textbf{Registered Consumers}\label{rc}} The entire community that is participating in getting meals from the students' mess. The employees are understood to be members of the registered consumers. The IMSc students', IMSc Faculty, IMSc Administration , IMSc cooking and cleaning staff opting for the students' mess are members of the registered consumers.
\item{\textbf{Consumers}} People who utilise the Students' Mess. The registered customers are a subset of this set of people. These would include people who didn't register themselves in the functioning of the students' mess.
 \item{\textbf{Tenure}} The duration during which the committee members would be active in trying to solve the Mess Problem. It is to be a maximum period of one month.
\item{\textbf{Rules}} The set of instructions that the Committee members in their tenure uphold. It is written in the guidelines section of this document.
\item{\textbf{GBM}} General body meeting of IMSc. A congregation of students and possibly faculty and others to discuss and strive to solve issues related to IMSc's functioning. IMSc's functioning also includes the student's mess operations.
\item{\textbf{Issues of Concern \label{ioc}}} The Mess Committee members might be unhappy with some of the existing rules or feel the need to add more rules during their tenure. These two scenarios are called issues of concern.
\item{\textbf{Consensus Mechanism}\label{cons}} A voting process in which  $\frac{2}{3}^{\textbf{rd}}$ majority is demanded from registered customers on the issues of concern. It is supposed to be attended by members of the registered customers.
\item{\textbf{Modification}\label{mod}} If the Mess Committee have Issues of Concern regarding the Constitution, they can activate the consensus mechanism.  \textbf{Only when} the demand for majority is met can the mess committee go ahead with the ratification of the alterations. Thus this will lead to an updated Constitution and we subsequently rename this document with the version number being increased by 1.
\item{\textbf{Higher Authority}} If there is a student body member(/student body) who will regulate the activities of the mess committee he/she(/they) is(/are) supposed to be considered as a Higher Authority. Other Higher Authorities with higher priority of influence include faculties/administration staff in charge of overseeing the Mess Committee's operations. 

\item{\textbf{Solution}\label{sol}} The mess committee's algorithm or process as an answer to the Mess problem.

\item{\textbf{Upgrades} \label{upg}} Changes to the solutions of the Mess Problem by the mess committee. this can be done via the usual mechanism of GBM already existing in IMSc.

\item{\textbf{Supply Chain}} Suppliers from the market that will provide raw materials and goods needed by the mess committee to solve the Mess problem. 
\item{\textbf{Treasurer}} A member of the mess committee that has the entire understanding of the expenses and credit that the student mess problem is involved with.
\item{\textbf{Market Surveyor}} A member of the mess committee whose usual objective will be to optimise on the minimum cost amongst all possible options available in market and thereby fix the supply chain.  Or in different circumstances even a tastier and fresher solutions of the items bought, need not necessarily be cheaper. Basically the cost function that he/she has to optimise on is something that is a product of the cost of the items needed from the market by the students and subjective notions of fresh, good quality of items available.
\item{\textbf{Delivery Auditor} } A member of the mess committee who would be tasked with checking whether the items listed in the bill match with the actual physical items delivered by the supply chains. In case of mismatches there should be a record kept of these losses that should be public to everyone of the students. In case the committee partners up with a supply chain who are regularly involved in such practises it should be a good signal for the next committee formed to look for alternatives and blacklist such contractors. Also, here it should be noted that we should be flexible enough to give such contractors a chance in the future even though they got blacklisted by a prior generation of committee members. We should be open to the idea that humans have the potential to evolve and become better than their previous versions. 
\item{\textbf{Corruption}} This should be a list of unwanted acts, perhaps will get expanded with successive iterations of the mess committee formations. Currently, it is defined as an act of providing false data especially on the monetary aspects during the attempt at the solution to the mess problem. 
\end{itemize}
\section{Rules}
\begin{enumerate}
\item The committee members are supposed to be members from the \textbf{students of IMSc} who would either be \textbf{nominated} by the previous Committee members or \textbf{volunteer} by themselves. 
\item The nature of the mess committee is approximately \textbf{democratic} that is \textit{of the registered customers, by the students in the registered customers, for the registered customers}. 
\item The services of the students among the registered customers towards the mess committee responsibility is \textbf{mandatory} in nature. Everyone must fill in the shoes at least once in their academic tenure.
\item A student fulfilling the duty of the mess committee \textbf{cannot} hold the same position for at least $9$ \textbf{months} from the end of his/her tenure.  
\item The mess committee can either designate individual roles to aid in the solving of the Mess problem or share the roles among themselves according to their liking during their entire tenure.
\item Some key roles that the committee must have : \textbf{Treasurer} , \textbf{Market Surveyor}, \textbf{Delivery Auditor}. There might be other roles fulfilled by other members or shared amongst the members.
\item  The committee members should ensure that they do not discriminate amongst the minority palate of the students. That is to say that they should ensure recipes that cater to the people from all parts of India.
\item A suggested way to go about solving the Mess Problem will be to first take a survey of the food preferences from their registered customers and then matching it to market availability. Which would  possibly make the prices per plate vary for each instance of the Mess Committee's functioning.
\item The committee members are \textbf{eligible} for a discounted pricing of the food that they would eat from the canteen. They can choose to forfeit such advantages. The discount should depend on factors that would be better understood in upcoming generations of the student committee. For the first student's body,   a $5\%$ discount is allotted on the actual prices that a fellow registered customer who is not in the mess committee has to pay. The eligibility comes with a catch that is explained in the next point.
\item For issues of \textbf{accountability} of the students committee. At the end of the tenure a GBM involving as many registered customers as possible, would be organised to question whether the committee was able to do a good job. Based on negative feedback, if certain members of the mess committee were availing the discount, they will have to forfeit that credit towards their services. Otherwise the mess committee members will be duly rewarded for their services.
\item The following data should be \textbf{public information} and well maintained by students committee during their tenure. 
\begin{enumerate}
\item The solution concept proposed by the mess committee. It involves the automation employed. The entire code should be public to all among the registered customers.
\item The amount of food wastage per day.
\item The billing of food and general items procured by the mess committee members. 
\item Whosoever among the mess committee audits on a particular day would inform publicly that he/she has done the audit for the day.
\item The accounts of all registered customers on a per monthly basis.
\item The losses incurred if any from the supply chain per day.
\item People who couldn't avail the food in required quantities per day.
\item A complaint box which should be public as well so that others can know what has transpired.
\end{enumerate}

\item It is expected that the public maintaining of records will be able to curtail \textbf{corruption} possibilities. But, even then if a mess committee is found to be corrupt by the registered customers or people from the \textbf{Higher Authority} then a mechanism of dissolving the committee is proposed to the \textbf{Higher Authority} by calling a GBM. And further penalties for the corrupt committee members having to cough up $10\%$ extra on the food prices for a time period is suggested. But such decisions will be ultimately taken under the jurisdiction of the GBM.
\item Whatever \textbf{solution} the mess committee adopts it should understand its \textbf{pressure points} and have \textbf{backup systems} to take over, in case those pressure points fall apart. The mess committee  should always strive to  clearly outline these to the registered customers. An example of a pressure point will be : if say the information of certain form of meals were done via an online system, then the pressure point is the over reliance on that online system because it might fail due to some natural calamity, like a cyclone passing by. The mess committee members should be prepared for such exceptional cases as well.

\item Whenever an issue of concern has come up, the mess committee is supposed to involve all the registered customers in the decision process. And activate the consensus mechanism amongst the registered customers that show up.

\item Whenever an upgradation idea occurs to the mess committee they should call for a GBM informing the ideas to all the registered customers and discuss to crystallise a final version of the idea and take a call on the upgradation to the existing solution of the mess problem.
\item It is understood that ideas of upgradation can occur to other registered customers not within the mess committee as well. They should take it up in their turn as a member of the mess committee to successfully implement the idea via the same upgradation mechanism described in the previous point. 
\item The mess committee members should mediate between the \textbf{employees} and IMSc authorities when specific circumstances occur, like circumstances regarding the health of an employee or if an employee needs to get back to their family, etc. They can also pass the matter to the Higher Authorities and let them solve these issues.
\item The mess committee members should be mindful of their own labour exploitation tendencies. Suggestions of providing off-days to the employees can be taken up.

\item The mess committee members should strive to uphold good man management skills to deal with the \textbf{employees} and the textbf{registered customers}. 

\item Each mess committee should \textbf{strive} towards making the functioning of their \textbf{solution} clear to the immediate next generation of mess committee members. Assuming they can always know who the future members of the committee will be, the mess committee members at the current instance can give them a guided tour ( an internship of sorts) of the current functioning near about end of their tenure. 
\par The future mess committee members \textbf{can choose to preserve the similar functioning} or if they believe they have identified inefficiencies in the present ways they \textbf{can go ahead and try and improve the solution} by calling for \textbf{upgradations}.



\item In spite of the best efforts to have an honest functioning of the mess committee , the committee 
members might find themselves having to deal with corruption and red tape which is beyond their purview. Being empathetic and aware of this situation  would aid in collective good for all members of the registered customers. 

\item It is understood that not all students from the registered customers would fall under the enthusiastic spectrum of trying to solve the mess problem and hence the previous generation mess committee members upon foreseeing such situations and nominate a team which would allow for a smooth functioning of the future mess committee involving such student members.
\end{enumerate}
\end{document}